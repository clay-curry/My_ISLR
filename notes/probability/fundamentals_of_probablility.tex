% !TEX TS-program = pdflatex
% !TEX encoding = UTF-8 Unicode

\documentclass[11pt]{article} % use larger type; default would be 10pt
\usepackage[utf8]{inputenc} % set input encoding (not needed with XeLaTeX)
\usepackage{clays_notes}

\newcommand\R[1]{\mathbb{R}^{#1}}
\newcommand\C[1]{\mathbf{C^{#1}}}
\newcommand\F[1]{\mathbf{F^{#1}}}
\newcommand\U{\mathbf{U}}
\renewcommand\S{\mathcal{S}}
\renewcommand\u{\mathbf{u}}
\newcommand\V{\mathbf{V}}
\newcommand\W{\mathbf{W}}


% FONT STYLES

\title{Linear Algebra Done Right \\ Axler, Sheldon}
\author{Notes by:  \\ Clay Curry}
\date{}

\begin{document}

\section{Probability}
Probability is a number that quantifies the likelihood of a given event when it is not yet known whether the event will happen or not. This definition is circular because it uses the concept of likelihood, which is a synonym for probability. Nonetheless, we can use it as a starting point. It highlights two important facts:
\points{probablility refers to an event;}
{probablility is a number.}
\noindent
By elaborating on these two facts, we will give an (almost entirely) rigorous definition of probability. 

\subsection{Sample space, sample points and events}
\vspace{10 pt}

The first thing we do when we start thinking about the probability of an event is to list \textbf{a number of things that could possibly happen}. The things in this list form a set, which we denote by $\Omega$.

\definition{Experiment, Sample Space}
{An \textbf{experiment} or \textbf{trial} (see below) is any procedure that can be infinitely repeated and has a well-defined set of possible outcomes, known as the \textbf{sample space} and usually denoted by $\Omega$. For a set of outcomes $\Omega$ to be a sample space, $\Omega$ must satisfy the following points
\points{\textbf{Mutually exclusive outcomes.} \hspace{5 pt} Only one of the things in $\Omega$ can happen. That is, if $\omega \in \Omega$ happens, then none of the things in the set $\{ \bar \omega \in \Omega : \bar \omega \ne \omega \}$ can happen.}
{\textbf{Exhaustive outcomes.} \hspace{5 pt} At least one of the things in $\Omega$ will happen.}

}

Random experiments are often conducted repeatedly, so that the collective results may be subjected to statistical analysis. A fixed number of repetitions of the same experiment can be thought of as a composed experiment, in which case the individual repetitions are called \textbf{trials}. 

\definition{Sample point}
{An element $\omega \in \Omega$ is called a \textbf{sample point}, or possible outcome.}

\definition{Realized outcome}
{When we learn that $\omega \in \Omega$ has happened, $\omega$ is called the \textbf{realized outcome}.}

\definition{Event}
{A subset $E \subseteq \Omega$ is called an event. $E$ is a \textit{sigma algebra} on $\Omega$, which is defined below.}

\subsection{Space of Events}
\end{document}

















